\DescribeMacro{\drefcalc[\meta{lets}]\{\meta{expr}\}[\meta{format}]}

The \cmd{\drefcalc} command is the core function of calculating with
data points. It is based on the pgfmath engine. It uses the required
argument as an mathematical expression, but has additional features,
that can be used.

\begin{quote}
 \verb|\drefcalc{(4+7)/12 * 100}| $\Rightarrow$ \drefcalc{(4+7)/12 * 100}
\end{quote}

\noindent It adds support for the \texttt{data} function within pgfmath, that
references symbolic data points. The keyname has to be in double
quotes to indicate a string, but you can easily define an appropriate
macro, that abstracts from \verb|data("")|.


\begin{quote}
 \begin{verbatim}\drefcalc{data("/control group/recovered") /
     data("/control group/mice count")}\end{verbatim} 
   $\Rightarrow$ \drefcalc{data("/control group/recovered") / data("/control group/mice count")}\\
 \verb|\newcommand{\cg}[1]{data("/control group/#1")}|\newcommand{\cg}[1]{data("/control group/#1")}\\
 \verb|\drefcalc{\cg{recovered}/\cg{mice count}}| $\Rightarrow$ \drefcalc{\cg{recovered}/\cg{mice count}}
\end{quote}

\noindent The first optional argument lets you define constants within pgfmath
(zero arity functions, that can be called without parenthesis). Those
bindings are only valid for the current \cmd{\datarefcalc} call.

\begin{quote}
 \verb|\drefcalc[A=\cg{recovered},B=12*20]{B/A}| $\Rightarrow$ \drefcalc[A=data("/control group/recovered"),B=12*20]{B/A}\\
 \verb|\drefcalc[X=100]{30/X}| $\Rightarrow$ \drefcalc[X=100]{30/X}
\end{quote}

\noindent The second optional argument, that appears after the required argument does define the pgfmath's \texttt{number format}.
\begin{quote}
 \verb|\drefcalc{1/3}[precision=5,fixed]| $\Rightarrow$ \drefcalc{1/3}[fixed,precision=5]
\end{quote}

\DescribeMacro{\drefcalc*}
\DescribeMacro{\drefresult}
\DescribeMacro{\drefformat}

\noindent When you get confused of the which optional argument does
what, just think of a pipe. First you define bindings, then you
calculate, then you emit stuff. When drefcalc is called with an star
argument, it does not print the result, but does only set
\cmd{\drefresult}. \cmd{\drefformat} is used to format a number.

\begin{quote}
 \verb|\drefcalc*{1/3} ABC: \drefresult| $\Rightarrow$ \drefcalc*{1/3} ABC: \drefresult\\
 \verb|\drefformat[fixed,precision=1]{\drefresult}|$\Rightarrow$ \drefformat[fixed,precision=1]{\drefresult}\\
 \verb|\drefformat[sci]{100000}| $\Rightarrow$ \drefformat[sci]{100000}
\end{quote}

\DescribeMacro{\dreflet}

The bindings for \cmd{\drefcalc} are only local to that macro
call. Defining a binding for the current group can be done with
\cmd{\dreflet}.

\begin{quote}
  \verb|\newcommand{\cg}[1]{data("/control group/#1")}|\newcommand{\cg}[1]{data("/control group/#1")}\\
  \verb|\dreflet{percent=data("/med A/mice count")/100}|\dreflet{percent=data("/med A/mice count")/100}\\

  The result cleary shows that a lorem ipsum kills\\
  \verb|\drefcalc{\cg{dead after 24h}/percent}| percent within 24
  and \verb|\drefcalc{\cg{dead after 28h}/percent}| percent within 48 hours.

  The result cleary shows that a lorem ipsum kills
  \drefcalc{\cg{dead after 24h}/percent} percent within 24
  and \drefcalc{\cg{dead after 48h}/percent} percent within 48 hours.
\end{quote}

\DescribeMacro{\drefprojection}

\DescribeMacro{\drefrow}

\DescribeMacro{\drefusagereport}


\DescribeMacro{\drefassert}
