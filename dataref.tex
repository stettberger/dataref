% Copyright (c) 2013-2016 Christian Dietrich
%
% This work may be distributed and/or modified under the
% conditions of the LaTeX Project Public License, either version 1.3
% of this license or (at your option) any later version.
% The latest version of this license is in
%   http://www.latex-project.org/lppl.txt
% and version 1.3 or later is part of all distributions of LaTeX
% version 2005/12/01 or later.
%
% This work has the LPPL maintenance status `maintained'.
%
% The Current Maintainer of this work is Christian Dietrich
%
% This work consists of the files dataref.tex and dataref.sty

\documentclass{ltxdoc}
\usepackage[usagereport]{dataref}[2015/09/17]

\EnableCrossrefs
\CodelineIndex
\RecordChanges

\usepackage{listings}
\usepackage{pdfcomment}
\usepackage{filecontents}
\begin{filecontents}{datapoints.tex}
\drefset{/control group/mice race}{Black Six}
\drefset{/control group/mice count}{32}
\drefset{/control group/dead after 24h}{3}
\drefset{/control group/dead after 48h}{7}
\drefset{/control group/recovered}{6}

\drefset{/med A/mice race}{Black Six}
\drefset{/med A/mice count}{32}
\drefset{/med A/dead after 24h}{6}
\drefset{/med A/dead after 48h}{1}
\drefset{/med A/recovered}{9}

\drefsethelp{.*/mice race}{The mice race used for experiments heavily
     influences the outcome of the results}

\drefsethelp{.*/(dead after|recovered)}{Of all infected mice, a
  certain number died within a specified period of time. A certain
  recovered from the infection. Each mouse is in exactly one category.}
\end{filecontents}
\input{datapoints}
% \OnlyDescription
\drefkeys{prefix=/foo,value=123,save=/foo}
\drefset{/override test}{2}
\begin{document}
\drefsave{/override test}{4}
\drefkeys{prefix=}

\CheckSum{0}


 \CharacterTable
  {Upper-case    \A\B\C\D\E\F\G\H\I\J\K\L\M\N\O\P\Q\R\S\T\U\V\W\X\Y\Z
   Lower-case    \a\b\c\d\e\f\g\h\i\j\k\l\m\n\o\p\q\r\s\t\u\v\w\x\y\z
   Digits        \0\1\2\3\4\5\6\7\8\9
   Exclamation   \!     Double quote  \"     Hash (number) \#
   Dollar        \$     Percent       \%     Ampersand     \&
   Acute accent  \'     Left paren    \(     Right paren   \)
   Asterisk      \*     Plus          \+     Comma         \,
   Minus         \-     Point         \.     Solidus       \/
   Colon         \:     Semicolon     \;     Less than     \<
   Equals        \=     Greater than  \>     Question mark \?
   Commercial at \@     Left bracket  \[     Backslash     \\
   Right bracket \]     Circumflex    \^     Underscore    \_
   Grave accent  \`     Left brace    \{     Vertical bar  \|
   Right brace   \}     Tilde         \~}


 \changes{v0.4}{2015/04/21}{Remove Spurious Whitespaces}
 \changes{v0.1}{2013/12/06}{Initial version}

 \DoNotIndex{\newcommand,\newenvironment}

 \providecommand*{\url}{\texttt}
 \GetFileInfo{dataref.sty}
 \title{The \textsf{dataref} package}
 \author{Christian Dietrich 2013-2014\\ \url{stettberger@dokucode.de}\\
        \url{https://github.com/stettberger/dataref}}
 \date{\filedate~\fileversion}

 \maketitle

 \section{Introduction}

 Writing scientific texts is a craft. It is the craft of communicating your results to your
 colleagues and to the curious world public. Often your conclusions are based upon facts and
 numbers that you gathered during your research for the specific topic. You might have done many
 experiments and produced lot of data. The craft of writing is to guide your reader through a
 narrative that is based upon that data. But there may be many versions of that data. Perhaps you
 found a problem in your experiment, while already writing, that forces you back into the
 laboratory. After a while, the moon has done its circle many times, you return from that dark
 place and your methodology has improved as significantly as your data has. But now you have to
 rewrite that parts of the data that reference the old data points.

 The \textsf{dataref} is here to help you with managing your data points. It provides you with
 macro style keys that represent symbolic names for your data points. You can reference those
 symbolic names with \cmd{\dref}, use them in calculations to have always up-to-date percentage
 values, define projections between sets of data points and document them. \textsf{dataref} also
 introduces the notion of assertions (\cmd{\drefassert}) for your results to ensure that your prosa
 text references fit the underlying data.

 \section{Usage (or \dref*{/control group/mice count} mice)}

 The \textsf{dataref} package heavily uses \textsf{pgfkeys} and \textsf{pgfmath} to perform storage
 and operations upon data points. See \texttt{texdoc pgfmanual} for further information about those
 topics.

 \DescribeMacro{\drefset\{\meta{name}\}\{\meta{value}\}}

 The \cmd{\drefset} command is used to define the symbolic data points. The first argument is the
 symbolic name, the second argument is the value. The value can be a number, but it can also be
 arbitrary text. The key may contain virtually all characters, including spaces and slashes. It is
 good practice to use a hierarchy to structure you data point names.

 \lstinputlisting[language=tex, firstline=5, lastline=15,basicstyle=\footnotesize,frame=single]{datapoints.tex}

 The code snippet, which is best stored in an external file, and which might be auto-generated, is
 best read with \cmd{\input}. It defines 10 symbolic names, that are partitioned into two
 "directories" (\texttt{control group} and \texttt{medicament A}).

 \DescribeMacro{\dref*\{\meta{name}\}}
 \DescribeMacro{\dref[\meta{format}]\{\meta{name}\}}


 This macro is used to reference a single symbolic data point. The value stored in that datapoint
 is inserted into the text. \cmd{\dref} additionally marks the data point as used. It will then
 appear in the dref usage report. For undefined keys the default behaviour is to abort the
 compilation. But the package option \texttt{ignoremissing} just outputs a warning. All
 referenced/missing/found datapoints are noted in the aux file.

 \vspace{1em}\noindent\begin{tabular}{ll}\toprule
 Macro & Expansion \\
 \midrule
 \cmd{\dref*\{/control group/mice race\}} & \dref*{/control group/mice race} \\
 \cmd{\dref*\{/control group/mice count\}} & \dref*{/control group/mice count} \\
 \cmd{\dref[sci,precision=2,zerofill=true]\{/med A/recovered\}} & \dref[sci,precision=2,zerofill=true]{/med A/recovered} \\ \bottomrule
 \end{tabular}\vspace{1em}

 \cmd{\dref} additionally takes an optional argument. This argument is interpreted as
 \texttt{/pgf/number format/} argument. See the pgf/tikz manual for more information. Only in the
 unstarred version this macro parses the value as a number. Be aware that
 \cmd{\dref} is \textbf{not} expandable.

 \DescribeMacro{\drefvalueof\{\meta{name}\}}

 Since \cmd{\dref} is not expandable, this macro can be used to get the bare value of a symbolic data point. But use it with caution, since it bypasses all internal book keeping.
 \begin{quote}
  \verb|\drefvalueof{/med A/mice race}| $\Rightarrow$ '\drefvalueof{/med A/mice race}'
 \end{quote}

 \DescribeMacro{\drefref\{\meta{name}\}}

 This is complement of \cmd{\drefvalueof}, it does \emph{only} the book keeping for a key (marking it as referenced etc.) So it might be used to compensate the usage of its bad sibling.
 \begin{quote}
  \verb|\drefref{/med A/mice race}|
 \end{quote}

 \DescribeMacro{[ignoremissing]}
 \DescribeMacro{[defaultvalue=1.0]}
 These two package options influence the behaviour regarding unknown
 keys. With \verb|ignoremissing| each missing symbolic datapoint is replaced by the defaultvalue. This behaviour might be useful when you use the .aux file, where the unknown keys are noted to extract data points from a third source (e.g. database, wikidata, etc). In the future a secondary tool will be provided to resolve those references.

 \DescribeMacro{\drefsethelp\{\meta{pattern}\}\{\meta{text}\}}
 \DescribeMacro{\drefhelp\{\meta{name}\}}

 \textsf{dataref} comes with a simple method for defining documentation for data points. This help can for example be used to communicate what is the concrete semantics of the data point. This is of special interest when writter and data gatherer are not the same person. \cmd{\drefsethelp} takes two arguments: first a regular expression that matches the symbolic data point, second the help text.

 \lstinputlisting[language=tex, firstline=17, lastline=18,basicstyle=\footnotesize,frame=single]{datapoints.tex}

 The helptext for a key is obtained by using the \cmd{\drefhelp} macro. It checks all defined helps (in linear order, first defined, first matched), and prints the first matching help text.

 For \textbf{LuaTeX}: When using this \TeX{} engine, only Lua (string.find)
 regular expressions are supported as patterns.
 \begin{quote}
  \verb|\drefhelp{/med A/mice race}|
 \end{quote}

 \DescribeMacro{\drefcalc[\meta{format}]\{\meta{expr}\}}
 \DescribeMacro{data("\meta{key}")}
 \DescribeMacro{d(\meta{key})}

 The \cmd{\drefcalc} command is the core function of calculating with data points. It is based on
 the pgfmath engine. It uses the required argument as a mathematical expression, but has additional
 features, that can be used.

 \begin{quote}
  \verb|\drefcalc{(4+7)/12 * 100}| $\Rightarrow$ \drefcalc{(4+7)/12 * 100}
 \end{quote}

 \noindent It adds support for the \texttt{data} function within pgfmath, which
 references symbolic data points. The keyname has to be in double
 quotes to indicate a string, but you can easily define an appropriate
 macro that abstracts from \verb|data("")|. As a quote-free
 alternative to the data command, \cmd{\drefcalc} provides also \verb|d(<key>|).



\begin{quote}
  \verb|\drefcalc{data("/med A/mice count") * 100}| $\Rightarrow$ \drefcalc{data("/med A/mice count") * 100}

  \verb|\drefcalc{d(/med A/mice count) * 100}| $\Rightarrow$ \drefcalc{d(/med A/mice count) * 100}
\end{quote}


 \noindent The optional argument lets you give a number format, which is used for printing the
 result number (\verb|/pgf/number format|).

 \begin{quote}
  \verb|\drefcalc[precision=5,fixed]{1/3}| $\Rightarrow$ \drefcalc[fixed,precision=5]{1/3}
 \end{quote}

 \cmd{\drefcalc} works as well in a \textbf{/pgf/fpu} environment or a normal one. The FPU feature of
 pgfmath is used to handle large numbers, which may occur often when handling experiment data
 points.

 \begin{quote}

 \end{quote}
 \dreflet{A=123456789,B=987654321, a=12, b=98}



 \begin{tabular}{lrr}
 \multicolumn{3}{l}{\cmd{\dreflet\{A=123456789, B=987654321, a=12, b=98\}}} \\
\toprule
 Macro & Inserted Text & \cmd{\drefresult} \\\midrule
 \verb|\drefcalc[/pgf/fpu]{A/B}| & \drefcalc[/pgf/fpu]{A/B} & \drefresult\\
 \verb|\drefcalc{a/b}| & \drefcalc[]{a/b} & \drefresult \\
 \verb|\drefcalc*[/pgf/fpu]{A/B}| & \drefcalc*[/pgf/fpu]{A/B} & \drefresult\\
 \verb|\drefcalc*{a/b}| & \drefcalc*[]{a/b} & \drefresult \\
 \bottomrule
 \end{tabular}


 \DescribeMacro{\drefcalc*}
 \DescribeMacro{\drefresult}
 \DescribeMacro{\drefformat\{\meta{number}\}}

 \begin{quote}
  \verb|\drefcalc*{1/3} ABC: \drefresult| $\Rightarrow$ \drefcalc*{1/3} ABC: \drefresult\\
  \verb|\drefformat[fixed,precision=1]{\drefresult}|$\Rightarrow$ \drefformat[fixed,precision=1]{\drefresult}\\
  \verb|\drefformat[sci]{100000}| $\Rightarrow$ \drefformat[sci]{100000}
 \end{quote}

 \DescribeMacro{/dref/let=\{\meta{lets}\}}
 \DescribeMacro{\dreflet\{\meta{lets}\}}

 Since symbolic key names can get long, dataref has the possibility to define variables for use
 within mathematical expression from other expressions. These "let"-bindings can either be defined
 locally for a \cmd{\drefcalc} commando with a pgf key or globally with \cmd{\dreflet}.

 \begin{quote}
  \verb|\drefcalc[/dref/let={A=12*20,B=\cg{recovered}}]{A/B}| $\Rightarrow$ \drefcalc[/dref/let={B=data("/control group/recovered"),A=12*20}]{A/B}\\
  \verb|\drefcalc[/dref/let={X=100}]{30/X}| $\Rightarrow$ \drefcalc[/dref/let={X=100}]{30/X}
 \end{quote}

 The bindings for \cmd{\drefcalc} are only local to that macro
 call. Defining a binding for the current group can be done with
 \cmd{\dreflet}.


 \begin{quote}
   \verb|\newcommand{\cg}[1]{data("/control group/#1")}|\newcommand{\cg}[1]{data("/control group/#1")}\\
   \verb|\dreflet{percent=data("/med A/mice count")/100}|\dreflet{percent=data("/med A/mice count")/100}\\

   The result clearly shows that a lorem ipsum kills\\
   \verb|\drefcalc{\cg{dead after 24h}/percent}| percent within 24
   and \verb|\drefcalc{\cg{dead after 28h}/percent}| percent within 48 hours.

   The result cleary shows that a lorem ipsum kills
   \drefcalc{\cg{dead after 24h}/percent} percent within 24
   and \drefcalc{\cg{dead after 48h}/percent} percent within 48 hours.
 \end{quote}




 \DescribeMacro{\drefrel*[\meta{opts}]\{\meta{key}\}}
 \DescribeMacro{\drefrel[\meta{opts}]\{\meta{key}\}}

 The \cmd{\drefrel} macro is used to calculate relations between a base value and a concrete key. A
 prominent example of such a relation is the percent relation. \cmd{\drefrel} allows you to write
 down intentionally what relation you want to express without thinking about a concrete
 formula. The starred version of this macro does not print anything, but sets only
 \cmd{\drefresult}.

 \begin{quote}
 \noindent\verb|\drefrel[base=/med A/mice count,factor]{/med A/recovered}| \\
 \mbox{}\hspace{1cm} $\Rightarrow$ \drefrel[base=/med A/mice
 count,factor,percent]{/med A/recovered} \\
 \end{quote}

 The type of relation can be manipulated with various keys. Almost
 always the given argument key will be set in relation to a base
 value. The type of relation can be given as well as post-processing
 steps.

 Like \cmd{\drefcalc}, \cmd{\drefrel} sets the \cmd{\drefresult}
 macro accordingly.

 \DescribeMacro{/dref/base}
 \DescribeMacro{/dref/base plain}
 \DescribeMacro{/dref/value plain}

 This specifies the key that will be used as a base. Without the
 \textbf{base plain} option, the value will be interpreted as a symbolic
 datapoint. With the option, base contains the plain value. When
 \textbf{value plain} is given, the mandatory argument is interpreted as
 a number and not as a symbolic name.

 \verb|\drefrel[factor,base=50,base plain]{/med A/mice count}| $\Rightarrow$
 \drefrel[factor,base=50,base plain]{/med A/mice count}

 \verb|\drefrel[factor,base=50,base plain,value plain]{45}| $\Rightarrow$
 \drefrel[factor,base=50,base plain,value plain]{45}

 \DescribeMacro{/dref/factor}

 Is a base relation type, which cannot be mixed with other relation
 types. It simply divides the given value by the base value.

 \[\cmd{\drefresult} = \frac{\texttt{value}}{\texttt{base}}\]

 \DescribeMacro{/dref/increase}
 \DescribeMacro{/dref/overhead}

 Is a base relation type. It calculates the overhead factor a value
 show toward the base value. \texttt{increase} and  \texttt{overhead}
 are synonyms.

 \[\cmd{\drefresult} = \frac{\texttt{value} -
 \texttt{base}}{\texttt{base}}\]

 \verb|\drefrel[overhead,base=50,base plain,value plain]{45}| $\Rightarrow$
 \drefrel[overhead,base=50,base plain,value plain]{45}


 \DescribeMacro{/dref/delta}

 Is a base relation type. It calculates the difference between value
 and base.

 \[\cmd{\drefresult} = \texttt{value} - \texttt{base}\]

 \verb|\drefrel[delta,base=50,base plain,value plain]{45}| $\Rightarrow$
 \drefrel[delta,base=50,base plain,value plain]{45}

 \DescribeMacro{/dref/scale}
 \DescribeMacro{/dref/product}

 Is a base-relation type. It calculates the product of value and base.

 \[\cmd{\drefresult} = \texttt{value} \cdot \texttt{base}\]

 \verb|\drefrel[scale,base=50,base plain,value plain]{45}| $\Rightarrow$
 \drefrel[scale,base=100,base plain,value plain]{0.45}

 \DescribeMacro{/dref/percent}

 Is a post-processing type. It calculates the percent value from a fraction.

 \[\cmd{\drefresult} = \cmd{\drefresult} \cdot 100.0 \]

 \noindent\verb|\drefrel[factor,percent,base=/med A/mice count]{/med A/recovered}| \\
 \mbox{}\hspace{1cm}$\Rightarrow$ \drefrel[factor,percent,base=/med A/mice count]{/med A/recovered}

  \DescribeMacro{/dref/abs}

 Is a post-processing type. It takes the absolute value.

 \[\cmd{\drefresult} = \mid \cmd{\drefresult} \mid \]
 \verb|\drefrel[overhead,abs,base=50,base plain,value plain]{45}| $\Rightarrow$
 \drefrel[overhead,abs,base=50,base plain,value plain]{45}

  \DescribeMacro{/dref/negate}

 Is a post-processing type. It negates the value.

 \[\cmd{\drefresult} = \cmd{\drefresult} \cdot -1.0 \]

 \noindent\verb|\drefrel[factor,negate,base=/med A/mice count]{/med A/recovered}| \\
 \mbox{}\hspace{1cm}$\Rightarrow$ \drefrel[factor,negate,base=/med A/mice count]{/med A/recovered}

 \DescribeMacro{/dref/divide}

 Is a post-processing type. Divides the result by a contant
 factor. The argument must be a plain number.

 \[\cmd{\drefresult} = \cmd{\drefresult} \cdot \{divide\} \]

 \noindent\verb|\drefrel[value plain,divide=1e6]{1453342654}|
 $\Rightarrow$
 \drefrel[value plain,divide=1e6]{1453342654}


 \DescribeMacro{\drefprojection\{\meta{from}\}\{\meta{to}\}\{\meta{projection}\}}

 Sometimes one or multiple sets of data have to be projected/mixed into a
 new set of data that is fully dependent on those values. This is
 achieved with \cmd{\drefprojection}. It projects one data set (subdirectoy) into
 another one. Tithin the projection three different operations are
 possible: \cmd{\id}, \cmd{\rename} and \cmd{\calc}.

 \drefprojection{/control group}{/projection}{
       \id{mice race} % identity function
       \rename{mice count}{count} % renaming of points
       \calc{data("/dead after 24h")+data("/dead after 48h")}{died}
 }

 \begin{quote}
   \begin{verbatim}\drefprojection{/control group}{/projection}{
       \id{mice race} % identity function
       \rename{mice count}{count} % renaming of points
       \calc{data("/dead after 24h")+data("/dead after 48h")}{died}
     }\end{verbatim}

   \verb|\dref*{/projection/died}| $\Rightarrow$ \dref*{/projection/died}\\
   \verb|\dref*{/projection/mice race}| $\Rightarrow$ \dref*{/projection/mice race}\\
   \verb|\dref{/projection/count}| $\Rightarrow$ \dref{/projection/count}
 \end{quote}

 \DescribeMacro{\drefrow\{\meta{list}\}\{\meta{macro}\}}
 \DescribeMacro{\drefrow*}


 Often different columns in a table have to be obtained from your data
 points. Often those rows and columns are similar. Generating parts of
 tables within \LaTeX is very tricky, so \textsf{dataref} provides you
 with \cmd{\drefrow}. This macro iterates over a comma-separated list
 of values and fills out a macro which is interpreted as a symbolic
 data point. The entries are seperated with \& and printed. In the
 starred variant the resulting text is not interpreted as symbolic
 name, but as a macro. The symbolic name is expanded with \cmd{\drefvalueof}.

 The second argument is the macro, and can have two macro
 replacements. The first replacement \verb|#1| is the value of the list
 item, the second \verb|#2| is the index in the list.

 \begin{verbatim}\begin{tabular}{lccc}
   Group & $<$ 24h & $<$48h & recovered\\ \hline
   Control Group & \drefrow{dead after 24h,dead after 48h,recovered}%
                           {/control group/#1}\\
   Medicament A & \drefrow{dead after 24h,dead after 48h,recovered}%
                           {/med A/#1}\\
   Starred Variant & \drefrow*{B,C,D}{\#1=#1,\#2=#2}\\
 \end{tabular}
 \end{verbatim}


 \vspace{1em}\begin{tabular}{lccc}
   Group & $<$ 24h & $<$48h & recovered\\ \hline
   Control Group & \drefrow{dead after 24h,dead after 48h,recovered}%
                           {/control group/#1}\\
   Medicament A & \drefrow{dead after 24h,dead after 48h,recovered}%
                           {/med A/#1}\\\hline
   Starred Variant & \drefrow*{B,C,D}{\#1=#1,\#2=#2}\\
 \end{tabular}\vspace{1em}

 \DescribeMacro{\drefassert\{\meta{expr}\}}
 \DescribeMacro{[noassert]}

 Sometimes the underlying data changes while you are writing. But what
 if your prose text relies on certain characteristics of the data. \cmd{\drefassert} uses a pgfmath
 expression that evaluates to \verb|true| or \verb|false|. When the
 assertion holds (\verb|true|) nothing happens, only a terminal message
 is printed. When it does not hold (\verb|false|) the compilation is aborted.

 \begin{verbatim}\drefassert{data("/control group/mice count") > 30}
 Of the more than thirty infected mice...
 \end{verbatim}
 \drefassert{data("/control group/mice count") > 30}
 \begingroup \pgfset{/pgf/fpu} \drefassert{data("/control group/mice
 count") > 30} \endgroup

 The \textbf{noassert} package options disables the latex abortion. In
 that case only a warning message is printed on the terminal.

 \DescribeMacro{[annotate=none]}
 \DescribeMacro{[annotate=footnote]}
 \DescribeMacro{[annotate=pdfcomment]}
 \DescribeMacro{\drefannotate\{\meta{style}\}}


 While writing a document it is desirable to know, what key is used,
 while writing the text and generating the document. Therefore
 \textsf{dataref} provides the possibility to annotate values. The
 default package option \textbf{none} disables this kind of
 annotation. The \textbf{pdfcomment} option uses pdf
 annotations. Be aware that those annotations work properlyy only on a few
 selected PDF readers\footnote{In doubt use Acrobat}. \cmd{\drefannotate} sets the annoation
 style for the current group.

 \begin{quote}
   \verb|\drefannotate{none}|\drefannotate{none}\\
   \dref*{/control group/mice race}, \dref{/control group/mice count}, \drefcalc{100/3}
 \end{quote}

 \begin{quote}
   \verb|\drefannotate{footnote}|\drefannotate{footnote}\\
   \dref*{/control group/mice race}, \dref{/control group/mice count}, \drefcalc{100/3}
 \end{quote}

 \begin{quote}
   \verb|\drefannotate{pdfcomment}|\drefannotate{pdfcomment}\\
   \dref*{/control group/mice race}, \dref{/control group/mice count}, \drefcalc{100/3}
 \end{quote}

 \DescribeMacro{\drefusagereport}
 \DescribeMacro{[usagereport]}
 \DescribeMacro{[refall]}

 With the \textbf{usagereport} package option enabled,
 \cmd{\drefusagereport} generates a usagereport of all referenced
 keys. The usage report groups the keys by the help texts. If the
 refall package option is given, all keys are marked as referenced.

 \section*{Datagraphy}
 \drefusagereport

\end{document}
